\documentclass{article}
\usepackage{tai}

\title{Research Notebook}
\author{Taisuke Yasuda}

\begin{document}

\maketitle
\tableofcontents
\newpage

\section{May 20, 2017}
\subsection{Progress}
\begin{itemize}
  \item set up github for project
  \item plotted histograms of first trials
  \item plotted scatter plots of first trials (for stationarity)
\end{itemize}

\subsection{Statistical Model}
Recall that our model of the process is
\[
  X = \sum_{j=1}^N X_j,\qquad Y_j = \Bernoulli(p_j),\qquad (X_j\mid Y_j = 1)\sim N(\mu_j,\sigma_j^2), \qquad (X_j\mid Y_j = 0) = 0
\]
where $N, \mu_j, \sigma_j^2, p_j$ are all unknown parameters of the model. The $X_j$ random variable models the response amplitude of a single contact of which there are $N$, and the $Y_j$ random variable models the release success of a single contact. We assume that all of the $X_j$ and the $Y_j$ are independent.

\subsubsection{Justification}
The additivity of the potential is justified by Petterson and Einevoll \cite{pettersen2008amplitude}. The use of a Gaussian distribution for individual contacts is justified by Magee and Cook \cite{magee2000somatic}.

\section{May 22, 2017}
\subsection{Progress}
\begin{itemize}
  \item derived point estimates for release probability, mean response amplitude, and response amplitude variance under the constant parameter models
\end{itemize}

\subsection{Point Estimation of Model Parameters}
\subsubsection{Estimations Under a Simplified Model}
We have that the expectation of the model is
\[
  \mathbb E[X] = \sum_{j=1}^N \mathbb E[X_j] = \sum_{j=1}^N \mu_j\cdot p_j
\]
and that the variance is
\[
  \Var[X] = \sum_{j=1}^N \Var[X_j] = \sum_{j=1}^N \sigma_j^2\cdot p_j.
\]
Also note that the failure rate of the model, i.e.\ the probability that all $N$ contacts fail to release, can be approximated by
\[
  \mathbb P[X = 0]\approx \prod_{j=1}^N (1-p_j)
\]
by assuming that a contact produces a positive response everytime it succeeds in releasing a vesicle. Thus, with simplifying assumptions that all the $\mu_j$, $\sigma_j^2$, and $p_j$ are the same across the $N$ points of contact, and by fixing a value of $N$, we may find a plugin estimator for $\hat p$ and method of moments estimators for $\hat\mu$ and $\hat\sigma^2$ that depend on $\hat p$. If we let $\overline X$ be the sample mean, $S^2$ be the sample variance, and $p_f$ be the sample failure rate, these estimates are given by
\[
  \hat p = 1 - \sqrt[N]{p_f}, \qquad \hat\mu = \frac{\overline X}{N\hat p}, \qquad \hat\sigma^2 = \frac{S^2}{N\hat p}.
\]

\subsubsection{EM Algorithm}
Let us return to the general case. Suppose that we treat the response amplitudes of the individual contacts, the $X_j$ from $1\leq j\leq N$, as latent variables. Then, the joint distribution of the latent and observed variables will be an exponential family, so EM algorithm should work very well.

\section{May 23, 2017}
\subsection{Goals}
\begin{itemize}
  \item how was the data collected? can we really justify our model?
  \item workout details of EM and implement
\end{itemize}

\subsection{Progress}
\begin{itemize}
  \item 
\end{itemize}

\bibliography{citations}
\bibliographystyle{plain}

\end{document}
