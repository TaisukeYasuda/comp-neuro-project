\documentclass{article}
\usepackage{tai}

\title{Research Notebook}
\author{Taisuke Yasuda}

\begin{document}

\maketitle
\tableofcontents
\newpage

\section{May 20, 2017}
\subsection{Progress}
\begin{itemize}
  \item set up github for project
  \item plotted histograms of first trials
  \item plotted scatter plots of first trials (for stationarity)
\end{itemize}

\subsection{Statistical Model}
Recall that our model of the process is
\[
  X = \sum_{j=1}^N X_j,\qquad Y_j = \Bernoulli(p_j),\qquad X_j =
  \begin{cases}
    N(\mu_j,\sigma_j^2) & Y_j = 1 \\
    0 & Y_j = 0
  \end{cases}
\]
where $N, \mu_j, \sigma_j^2, p_j$ are all unknown parameters of the model. The $X_j$ random variable models the response amplitude of a single contact of which there are $N$, and the $Y_j$ random variable models the release success of a single contact. The additivity of the potential is justified by Petterson and Einevoll \cite{pettersen2008amplitude}. Note

\bibliography{citations}
\bibliographystyle{plain}

\end{document}
