\documentclass{beamer}

\title{Statistical Analysis of SST-Pyramidal Post-synaptic Amplitude Distributions}
\author{Taisuke Yasuda}

\usepackage{graphicx}

\begin{document}

\begin{frame}
  \titlepage
\end{frame}

\begin{frame}
  \tableofcontents
  % Plan this out into sections and subsections
\end{frame}

%%%%%%%%%%%%%%%%%%%%%%%%%%%%%%%%%%%%%%%%%%%%%%%%%%%%%%%%%%%%%%%%%%%%%%%%%%%%%%%%
% Problem Set Up
%%%%%%%%%%%%%%%%%%%%%%%%%%%%%%%%%%%%%%%%%%%%%%%%%%%%%%%%%%%%%%%%%%%%%%%%%%%%%%%%

\begin{frame}{Biology of Post-synaptic Amplitudes}
  % Draw a picture of how post-synaptic amplitudes work
\end{frame}

\begin{frame}{Experimental Measurements}
  % Show what we measure
\end{frame}

\begin{frame}{Extracting Post-synaptic Response Amplitudes}
  % Show how to find the actual data, i.e. zoom in and find height
\end{frame}

\begin{frame}{Typical and Extreme Cases of Response Amplitudes}
  % Show how the cells fail, and how sometimes they fire with large amplitudes
\end{frame}

%%%%%%%%%%%%%%%%%%%%%%%%%%%%%%%%%%%%%%%%%%%%%%%%%%%%%%%%%%%%%%%%%%%%%%%%%%%%%%%%
% Classical Mathematical Models for the Distribution Don't Work
%%%%%%%%%%%%%%%%%%%%%%%%%%%%%%%%%%%%%%%%%%%%%%%%%%%%%%%%%%%%%%%%%%%%%%%%%%%%%%%%

\begin{frame}{Compound Binomial Distribution Model}
  % Explain the binomial distribution model in picture and with equations
\end{frame}

\begin{frame}{Binomial Distribution Model Applied to Neuromuscular Junctions}
  % Show how binomial distribution model works in classical papers
\end{frame}

\begin{frame}{Post-synaptic Amplitude Distribution Over Trials}
  % Histogram showing high failure rate
\end{frame}

\begin{frame}{Inference of Parameters Assuming the Binomial Model}
  % Explain method of moments extraction of parameters
\end{frame}

\begin{frame}{Inference of Parameters Assuming the Binomial Model}
  % Show that inference of parameters works if it truly comes from this model
\end{frame}

\begin{frame}{Comparison of Simulated Model Against the Data}
  % Histogram of simulation vs original data
\end{frame}

\begin{frame}{Comparison of Simulated Model Against the Data}
  % Probabilities against N
\end{frame}


\end{document}
